%\documentclass{tufte-handout} 
\documentclass[11pt]{report}
\usepackage{pgfplots}
\pgfplotsset{compat=1.18}
\usepackage{standalone}

% Load your custom package
\usepackage{mcgillnotes2}

%----------------------------------------------------------------------------------------
%   TITLE SECTION
%----------------------------------------------------------------------------------------
\title{ 
    \normalfont\normalsize 
    {McGill University - Winter 2026} \\ [0pt] 
    \huge Lecture Notes - MATH 455
}
\author{Charles Zitella} 
\date{\vspace{-5pt}\normalsize\today} 

%----------------------------------------------------------------------------------------
%   DOCUMENT CONTENT
%----------------------------------------------------------------------------------------
\begin{document}
\maketitle
\tableofcontents

\section{Abstract Metric and Topological Spaces}
\subsection{Metric Spaces Review }
Throuhgout, assume $X$ is a non empty set.
\begin{defn}
    (Metric): $p : X \times X \to \R$ is called a \textit{metric}, and thus 
    $(X, p)$ a metric space, if for all $x,y,z \in X$
    \begin{itemize}
        \item $p(x,y) \geq 0$, 
        \item $p(x,y) = 0 \iff x = y$,
        \item $p(x,y) = p(y,x)$,
        \item $p(x,y) \leq p(x,z) + p(z,y)$ (Triangle Inequality).   
    \end{itemize}
\end{defn}

\begin{defn}
    (Norm): Let $X$ be a vector space.\footnote{closed under linear combinations} 
    A function $\| \cdot \| : X \to [0,\infty)$ 
    is called a \textit{norm}, and thus $(X, \| . \| )$ a \textit{normed vector space},
     if for all $u,v \in X$ and $\alpha \in \R$
    \begin{itemize}
        \item $\| u \| = 0 \iff u = 0$,
        \item $\| u + v \| \leq \| u \| + \| v \| $,
        \item $\| \alpha u \| = \left| \alpha \right|  \| u \| $.
    \end{itemize}
\end{defn}

\begin{rem}
    A norm induces a metric by $p(x,y) := \| x-y \| $.
\end{rem}

\begin{exmp} Examples of normed vector spaces:
    \begin{enumerate}
        \item $(\R^n, \left| . \right| )$ where $\left| x \right| = ({x_1}^2 + \dots {x_n}^2)$
        \item $L^p(E)$ for $E \subseteq \R^n, 1 \leq p \leq \infty $ where $\| f \|_{L^P(E)} =  
        (\int_{E}^{}\left| f(x) \right|^p dx)^{\frac{1}{p}}$ 
        \item Discrete metric: if $X$ is a non empty set, then $p(x,y) = \begin{cases}
            0 & x = y \\
            1 & x \neq y
        \end{cases} $
        \item $C([a, b]) = \left\{ f : [a,b] \to \R \mid f \text{ is continuous on }  [a,b] \right\} $ 
        for $a, b \subseteq \R$. Then, $\| f \|_{\infty} := \sup_{x \in [a,b]} \left| f(x) \right| =
        \max_{x \in [a,b]} \left| f(x) \right|$,  $p(f,g) = \| f - g \|_{\infty} $
    \end{enumerate}
\end{exmp}

\begin{defn}
    Given two metrics $p, \sigma$ on $X$, we say they are \textit{equivalent} if
    $\exists$ a $C > 0$ such that 
    $\frac{1}{C}\sigma(x,y) \leq p(x,y) \leq C\sigma(x,y)$ for every $x,y \in X$.
    A similar definition follows for equivalence of norms.     
\end{defn}

\noindent Given a metric space $(X, P)$, then, we have the notion of
\begin{itemize}
    \item open balls $B(x,r) = \left\{ y \in X : p(x,y) \leq r \right\} $ 
    \item open sets (subsets of $X$ with the property that for every $x \in X$, there is a constant
    $r > 0$ such that $B(x,r) \subseteq X$), closed sets, closures, and
    \item \textit{convergence} 
\end{itemize}

\begin{defn}[Convergence]
    $\left\{ x_n \right\}_{n = 1}^{\infty} \subseteq X $ converges to $x$ in $(X, p)$ if 
    $\lim_{n \to \infty} p(x_n, x) = 0$  
\end{defn}

\noindent We have several (equivalent) notions, then, of continuity; via sequences, 
$\epsilon \ \text{--} \ \delta$ definition, and by pullbacks (inverse images of open sets are open).

\begin{defn}[Uniform Continuity]
    $f : (X,p) \to (\R, \left| . \right| )$ uniformly continuous if $f$ has a 
    "modulus of continuity", i.e. there is a continuous function 
    $\omega : [0, \infty) \to [0, \infty)$ such that $\lim_{t \to 0^{+}} \omega(t) = 0 $, and 
    \[
        \left| f(x) - f(y) \right|  \leq \omega(p(x, y))
    \]  
    for every $x, y \in X$ 
\end{defn}
\begin{rem}
    For instance, we say $f$ Lipschitz continuous if there is a constant 
    $C > 0$ such that $\omega(.) = C (.)$. Let $\alpha \in (0,1)$. 
    We say $f$ $\alpha$-Holder continuous if $\omega(.) = C(.)^\alpha$ for some constant $C$.      
\end{rem}
\begin{defn}[Completeness]
    We say $(X, p)$ \textit{complete} if every Cauchy sequence in $(X, p)$ converges
    to a point in $X$. 
\end{defn}

\begin{rem}
    let $E \subseteq X$ and $(X, p)$ complete metric space. Then $(E, p)$ is complete iff
    $E \subseteq X$ is closed (so limits belong to E) 
\end{rem}  

\subsection{Compactness, Separability}

\begin{defn}[Open Cover, Compactness]
    $\left\{ X_\lambda \right\}_{\lambda \in \Lambda} \subseteq 2^X$
    \footnote{$2^X$ denotes the power set of $X$, i.e. the set of all subsets of $X$.},
    where $X_\lambda$ open in $X$ and $\Lambda$ an arbitrary index set, an \textit{open cover} 
    of $X$ if for every $x \in X$, $\exists \lambda \in \Lambda$ such that $x \in X_\lambda$.
    \footnote{A cover is finite if $\left| \Lambda \right| < \infty$ }
    $X$ is \textit{compact} if every open cover of $X$ admits a finite subcover. 
    We say $E \subseteq X$ compact if $(E, p)$ compact.  
\end{defn}

\begin{rem}
    for $E \subseteq X$, $X_\lambda \subseteq E$ is open in $(E, p)$ iff $X_\lambda$ is open in $(X, p)$
    Therefore, $E \subseteq X$ is compact iff every open cover of $E$ (in $X$) has a finite subcover.     
\end{rem}

\begin{rem}
    This definition leads to another definition of compactness 
    based on the finite intersection property.
\end{rem}

\noindent Useful consequence: if $(X, p)$ is compact metric space, and 
$\left\{ E_k \right\}_{k = 1}^{\infty} \subseteq X$ closed, and $E_{k + 1} \subseteq E_k \forall k$, 
$\cap_{k = 1}^{\infty} E_k \neq \emptyset$   

\begin{defn}[Totally Bounded, $\epsilon$-nets]
    $(X, p)$ is \textit{totally bounded} if 
    $\forall \epsilon > 0$, there is a finite cover of $X$ of balls with radius $\epsilon > 0$.
    \footnote{Totally bounded implies $(X, p)$ is bounded}
    If $E \subseteq X$, an $\epsilon$-net of $E$ is a collection 
    $\left\{ B(x_i, \epsilon) \right\}^N_{i=1} $ such that 
    $E \subseteq \bigcup^N_{i=1} B(x_i, \epsilon)$ and $x_i \in X$ 
    (note that $x_i$ need not be in $E$).
\end{defn}

\begin{defn}[Sequentially Compact]
    $(X, p)$ \textit{sequentially compact} if every sequence in $X$ has
    a convergent subsequence whose limit is in $X$. 
\end{defn}

\begin{defn}[Relatively/Pre-Compact]
    $E \subseteq X$ \textit{precompact} if $\overline{E}$ compact.   
\end{defn}

\begin{thm}
    TFAE:
    \begin{enumerate}
        \item $X$ complete and totally bounded;
        \item $X$ compact;
        \item $X$ sequentially compact.
    \end{enumerate}
\end{thm}

\begin{rem} TFAE:
    \begin{enumerate}
        \item $E$ is totally bdd and Cauchy Seq. converge
        \item $E$ is precompact
        \item $\forall \left\{ x_k \right\}_{k = 1}^{\infty} \subseteq E, \exists$ a convergent subsequence   
    \end{enumerate}

\end{rem}  

\noindent Let $f: (X, p) \to (\R, \left| . \right| )$ continuous with $(X, p)$ compact. Then,
\begin{itemize}
    \item $f(X)$ compact in $(\R, \left| . \right| )$;
    \item The max and min of $f$ over $X$ are attained;  
    \item $f$ is uniformly continuous. 
\end{itemize}

\begin{lem}
    Any cauchy sequence \footnote{$\forall \epsilon > 0, \exists N > 0 $ s.t. $\forall m, n > N$,
    \newline $\| x_n - x_m \| < \epsilon$   }
     converges iff it has a convergent subsequence.
\end{lem}
\begin{comment}
\begin{proof} \mbox{}\\
    $\Rightarrow$ trivial. 
    \newline
    $\Leftarrow$ Let $\left\{ x_n \right\}_{n \in \N}$ be cauchy in a metric space $(X, p)$ with
    convergent subsequence $\left\{ x_{n_k} \right\}_{k \in \N}$ which converges to some $x \in X$.
    Fix $\epsilon > 0$. Let $N \geq 1$ be such that for all 
    $m, n \geq N$, $p(x_n, x_m) < \frac{\epsilon}{2}$. Let $K \geq 1$ be such that for all 
    $k \geq K$, $p(x_{n_k}, x) < \frac{\epsilon}{2}$. Let $n, n_k \geq \max{\left\{ N, K \right\} }$, then       
    \[
        p(x_n, x) \leq p(x_n, x_{n_{k}}) + p(x_{n_{k}}, x) < \frac{\epsilon}{2} + \frac{\epsilon}{2} = \epsilon
    \] 
    Therefore, $\lim_{n \to \infty} x_n = x$.
\end{proof}
\end{comment}
\begin{proof} \mbox{}\\
    $\Rightarrow$ 
    If $\{f_{n}\}_{n=1}^{\infty} $ converges, then $\exists f : X \to \R$ s.t. $
    \| f_n - f \|_{\infty} \to 0 $, so all subsequences also converge to $f$. 
    \newline $\Leftarrow$  
    Now assume $\exists$ a subsequence $\{f_{n_k}\}_{k=1}^{\infty} \subseteq C(X)$ s.t. 
    $\lim_{k \to \infty} f_{n_k} = f$ in $C(X) \iff \| f_{n_k} - f \|_{\infty} \to 0$.      
    Suppose for the purpose of contradiction that $f_n \not\to f$. Thus, $\exists \epsilon > 0$, 
    and a subsequence $\{f_{n_j}\}_{j=1}^{\infty} \subseteq C(X)$ s.t. $\| f_{n_j} -f  \|_{\infty} 
    > \epsilon$ for every $j \geq 1$. Then, \[ \| f_{n_k} - f_{n_j} \|_{\infty} \geq  \| f_{n_j} - f \|
    _{\infty} -\| f - f_{n_k} \|_{\infty} > \epsilon - \frac{\epsilon}{2} = \frac{\epsilon}{2}\] for k 
    sufficiently large and for $n_k, n_j$ large enough. But this violates $\{f_{n}\}_{n=1}^{\infty} $ 
    being cauchy. (Contradiction), so we must have $f_n \to f$ in $C(X)$.
\end{proof}

\noindent Let $C(X) := \left\{ f : X \in \R \mid f \text{ continuous} \right\} $ and 
$\| f \|_{\infty} := \max_{x \in X} \left| f(x) \right| $ the sup norm. Then,  

\begin{prop}
    Let $(X, p)$ compact. Then $(C(X), \| . \| _{\infty})$ is complete.
\end{prop}
\begin{comment}
\begin{proof}
    Let $\left\{ f_n \right\}_{n = 1}^{\infty} \subseteq C(X)$ Cauchy with respect to $\| . \| _{\infty}$. 
    Then, there exists a subsequence $\left\{ f_{n_k} \right\} $ such that for each $k \geq 1$,
    $\| f_{n_{k + 1}} - f_{n_k}\| \leq 2^{-k}$  
    \newline
    (To construct this subsequence, let $n_1 \geq 1$ be such that 
    $\| f_n - f_{n_1}\|_{\infty} < \frac{1}{2} $ for all $n \geq n_1$, which exists since 
    $\left\{ f_n  \right\} $ Cauchy. Then, for each $k \geq 1$, define inductively $n_{k + 1}$ 
    such that $n_{k + 1} > n_k$ and $\| f_n - f_{n_{k + 1}} \| _{\infty} < \frac{1}{2^{k + 1}}$ 
    for each $n \geq n_{k + 1}$. Then, for any $k \geq 1$, $\| f_{n_{k + 1}} - f_{n_k}\|_{\infty} < 2^{-k}$
    , since $n_{k + 1} > n_k$.  ) .
\end{proof}
\end{comment}
\begin{proof}
    let $\{f_{n}\}_{n=1}^{\infty} \subseteq C(X)$ be Cauchy. Fix $k \in \N$. By Cauchy defn, let 
    $\epsilon = 2^{-k}$, so $ \exists N_k$ sufficiently large s.t. $\| f_{N_k} - f_{{N_k} + 1} \|_{\infty}
    < 2^{-k} $. We can then choose $\{n_{k}\}_{k=1}^{\infty} $ s.t. $n_k \to \infty$ and
    $\| f_{n_k} - f_{n_{k + 1}} \| < 2^{-k} \quad \forall k \in \N$. Let $j \in \N$. Then
    \[\| f_{n_{k + j}} - f_{n_k} \|_{\infty} \leq \sum_{\ell=k}^{k + j - 1} 
    \| f_{n_{\ell + 1}} - f_{n_\ell} \|_{\infty} \leq \sum_{\ell=k}^{k + j - 1} 2^{-\ell} \leq 
    \sum_{\ell=k}^{\infty} 2^{-\ell} \xrightarrow[k \to \infty]{} 0\]
    In particular, $\forall x \in X$ fixed, let $c_k := f_{n_k}(x)$. Then $\left| c_{k + j} - c_k \right| 
    \leq \| f_{n_{k + j}} - f_{n_k} \|_{\infty} \to 0 \quad \forall j \in \N$.
    Thus $\{c_{k}\}_{k=1}^{\infty} \subseteq \R$ is cauchy, so by completeness of $\R$,
    $\exists \overline{c} \in \R$ s.t. $\lim_{k \to \infty} c_k = \overline{c} =: f(x)$
    Doing this $\forall x \in X$, we have 
    \begin{align*}
        \left| f_{n_k}(x) - f(x) \right| 
        &= \lim_{j \to \infty} \left| f_{n_k}(x) - f_{n_{k + j}}(x) \right| \\
        &\leq \lim_{j \to \infty} \| f_{n_k} - f_{n_{k + j}} \|_{\infty} \\
        &\leq \sum_{\ell=k}^{\infty} 2^{-\ell} \xrightarrow[k \to \infty]{} 0
    \end{align*}
    $\Rightarrow \| f_{n_k} - f \|_{\infty} = \sup_{x \in X} \left| f_{n_k}(x) - f(x) \right|  
    \xrightarrow[k \to \infty]{} 0$, so $f_{n_k} \to f$ in $C(X)$. 
    Finally, by the lemma this implies $f_n \to f$ in $C(X)$, so
    $(C(X), \| . \|_{\infty} )$ is complete.            
\end{proof}

\begin{defn}[Density/Separability]
    A set $D \subseteq X$ is called \textit{dense} in $(X, p)$ if for every
    \footnote{If $A$ dense in $X$, then $\overline{A}$ dense in $X$}  
    nonempty open subset $A \subseteq X$, $D \cap A \neq \emptyset$. We say that $X$ is 
    \textit{separable} if there is a countable dense subset $D \subseteq X$.    
\end{defn}

\begin{prop}
    If $X$ compact, then $X$ is separable
\end{prop}
\begin{proof}
    Since $X$ is compact, it is totally bounded. Therefore, for $n \in \N$, there is some
    $K_n$ and $\left\{ x_i^n \right\} \subseteq X$ such that 
    $X \subseteq \cup_{i = 1}^{K_n} B(x_i^n, \frac{1}{n})$. Then, 
    $D = \cup_{n = 1}^{\infty}\cup_{i = 1}^{K_n} \left\{ x_i^n \right\} $ countable and 
    dense in $X$ 
\end{proof}

\subsection{Arzelà-Ascoli}

Goal: Find suitable conditions for a sequence to have a convergent subsequence in 
$(C(X), \| . \|_{\infty}) $.

\begin{defn}[Equicontinuous]
    A family $\mathcal{F} \subseteq C(X)$ is called \textit{equicontinuous} at
    $x \in X$ if $\forall \epsilon > 0$ there exists a $\delta_x > 0$ such that 
    if $p(x, x') < \delta_x$ then $\left| f(x) - f(x') \right| < \epsilon $ for every $f \in \mathcal{F}$.
    $\mathcal{F}$ is pointwise equicontinuous on $X$ if $\mathcal{F}$ is equicontinuous at
    every point $x \in X$. \footnote{
        if $\left| \mathcal{F} \right|  < \infty$, then $\mathcal{F}$ is pointwise equicontinuous on $X$.   
    }      
\end{defn}

\begin{exmp}
    Fix $M > 0, [a,b] \subseteq \R$. $\mathcal{F} := \left\{ f \in C([a,b]) \cap C'((a,b))
    \mid \left| f' \right| \leq M \right\} $. By Mean Value Theorem, $\left| f(x) - f(y) \right|
    \leq \left| f'(x^*) \right| \left| x - y \right| \leq M \left| x -y \right|  $ for some 
    $x^* \in [x, y]$, so $\forall x \in [a,b]$ 
    if $\left| x - y \right| < \frac{\epsilon}{M} $ then $\left| f(x) - f(y) \right| < \epsilon, 
    \; \forall f \in \mathcal{F}$, therefore $\mathcal{F}$ is pointwise equicontinuous on $[a,b]$.     
\end{exmp}

\begin{exmp}
    Consider $f_n(x) := x^n$ on $[0,1]$. Then $\{f_{n}\}_{n=1}^{\infty} $ is non equicontinuous
    at $x = 1$. $f_n(1) = 1 \; \forall n$, but the threshold to be close to $f_n(1)$ is 
    not uniform on n.  

\end{exmp}

\begin{marginfigure}
    \centering
    \input{fn_graph.tex}
    \caption{The sequence $f_n(x)=x^n$ is not equicontinuous.}
\end{marginfigure}

\begin{defn}[Pointwise, Uniform Boundedness]
    $\left\{ f_n \right\}$ pointwise bounded if $\forall x \in X, 
    \exists M(x) > 0$ such that $\left| f_n(x) \right| \leq M(x) \ \forall n$, and uniformly bounded
    if such an M exists independent of $X$.    
\end{defn}

\begin{defn}[Uniform Equicontinuous]
    $\mathcal{F} \subseteq C(X)$ is uniformly equicontinuous on
    $X$ if $\forall \epsilon > 0 \; \exists \delta > 0$ s.t. $\forall x,y \in X$ if $p(x, y) < \delta$,
    then $\left| f(x) - f(y) \right| < \epsilon, \; \forall f \in \mathcal{F}$.      
\end{defn}

\begin{rem}
    $\mathcal{F} $ equicontinuous at x $\iff$ all $f \in \mathcal{F}$ share the same modulus of 
    continuity at $x$, i.e. $\exists \omega_x$ s.t. $\left| f(x) - f(y) \right| 
    \leq \omega_x \left| x - y \right|,\; \forall f \in \mathcal{F}$.      
\end{rem} 
\begin{prop}[Sufficient Conditions for Uniform Equicontinuity]
\begin{enumerate}
    \item $\mathcal{F} \subseteq C(X)$ is uniformly Lipschitz continuous, i.e. $\exists M > 0$
    s.t. $\left| f(x) - f(y) \right| \leq Mp(x, y) \; \forall f \in \mathcal{F}$;
    \item $\mathcal{F} \subseteq C(X) \cap C^1(X)$ has a uniform $L^\infty$ bound on the
    1st derivative (same as earlier example, by MVT);
    \item If $(X, p)$ is compact and $\mathcal{F} \subseteq C(X)$ is pointwise equicontinuous on 
    $X$ $\Rightarrow \mathcal{F}$ is uniformly equicontinuous (Homework).          
\end{enumerate}
\end{prop}

\begin{lem}[Arzelà-Ascoli Lemma]
    Let $X$ be separable and let $\{f_{n}\}_{n=1}^{\infty} \subseteq C(X)$
    be pointwise bounded and equicontinuous. Then, there is a function $f \subseteq C(X)$ 
    and a subsequence $\{f_{n_k}\}_{k=1}^{\infty} $ which converges pointwise to $f$ on all of $X$.  
\end{lem}

\begin{proof}
    Let $D = \{x_{j}\}_{j=1}^{\infty} \subseteq X$ be a countable dense subset of $X$. Since
    $\left\{ f_n \right\} $ is pointwise bounded, $\left\{ f_n(x_1) \right\} $ as a sequence of 
    real numbers is bounded and so by Bolzano-Weierstrass, there is a convergent 
    subsequence $\left\{ f_{n(1, k)}(x_1) \right\}_k $ that converges to some $a_1 \in \R$.
    Consider now $\left\{ f_{n(1, k)}(x_2) \right\}_k$, which is again a bounded sequence of $\R$
    and so has a convergent subsequence, call it $\left\{ f_{n(2, k)}(x_2) \right\}_k$, which
    converges to some $a_2 \in \R$. Note that
    $\left\{ f_{n(2, k)} \right\} \subseteq \left\{ f_{n(1, k)} \right\} $, so also 
    $f_{n(2, k)}(x_1) \to a_1$ as $k \to \infty$. We can repeat this procedure, producing
    a sequence of real numbers $\left\{ a_\ell \right\} $, and for each $j \in \N$
    a subsequence $\left\{ f_{n(j, k)}\right\}_k \subseteq \left\{ f_n \right\} $ such that
    $f_{n(j, k)}(x_\ell) \to a_\ell$ for each $1 \leq \ell \leq j$. Define then
    \[
    f : D \to \R, \quad f(x_j) := a_j
    \] 
    Consider now
    \[
    f_{n_k} := f_{n(k, k)}, \quad k \geq 1
    \]
    the "diagonal sequence", and remark that $f_{n_k}(x_j) \to a_j = f(x_j)$ as $k \to \infty$
    for every $j \geq 1$. Hence, $\left\{ f_{n_k} \right\}_k$ converges to $f$ on $D$, pointwise.
    
    We claim now that $\left\{ f_{n_k} \right\}_k$ converges on all of $X$ to some function
    $f : X \to \R$, pointwise. Put $g_k := f_{n_k}$ for notational convenience. Fix $x_0 \in X,
    \epsilon > 0$, and let $\delta_{x_0} > 0$ be such that if 
    $x \in X$ such that $p(x, x_0) < \delta_{x_0}$,
    $\left| g_k(x) - g_k(x_0) \right| < \frac{\epsilon}{3}$. Since $D$ is dense in $X$, 
    $\exists x_j \in D$ s.t. $p(x_j, x_0) < \delta_{x_0}$. Since $\left\{ g_k(x_j) \right\}_k$ 
    converges, it is thus Cauchy, and hence for every 
    $k, \ell \geq K$, $\left| g_k(x_j) - g_\ell(x_j) \right| < \frac{\epsilon}{3}$. Therefore,
    \[
        \left| g_k(x_0) - g_\ell(x_0) \right| \leq 
        \left| g_k(x_0) - g_k(x_j) \right| + \left| g_k(x_j) - g_\ell(x_j) \right| + 
        \left| g_\ell(x_j) - g_\ell(x_0) \right| < \epsilon 
    \]  
    And thus $\left\{ g_k(x_0) \right\}_{k}  $ Cauchy as a sequence in $\R$. Since $\R$ is complete,
    then $\left\{ g_k(x_0) \right\}_{k}$ also converges, to, say, $f(x_0) \in \R$. Since $x_0$ was
    arbitrary, this means there is some function $f : X \to \R$ such that $g_k \to f$ pointwise
    on $X$ as we aimed to show.    
\end{proof}

\begin{thm}[Arzelà-Ascoli Theorem]
    Let $X$ be compact and let $\{f_{n}\}_{n=1}^{\infty} \subseteq C(X)$
    be uniformly bounded and uniformly equiontinuous. Then, $\exists$ subseq 
    $\{f_{n_k}\}_{k=1}^{\infty} $  and $f \in C(X)$ s.t. $f_{n_k} \xrightarrow[k \to \infty]{} f$ 
    in $C(X)$ (i.e. uniformly)    
\end{thm}
\begin{comment}
\begin{proof}
    Since $(X, p)$ is compact, it is thus separable. Also, uniform bounded/equicontinuous implies
    pointwise bounded/equicontinuous. Therefore, by Arzelà-Ascoli lemma, $\exists f: X \to \R$
    and $\{f_{n_k}\}_{k=1}^{\infty} $ s.t. $f_{n_k} \to f$ pointwise in $X$.
    Let $g_k := f_{n_k}$. 
    \newline \underline{Claim:} $\{g_{k}\}_{k=1}^{\infty} $ is uniformly Cauchy
    \footnote{Cauchy sequence in $(C(X), \| . \|_{\infty} )$  }
    \newline Fix $\epsilon > 0$. By uniform equicontinuity, $\exists \delta > 0$ s.t. 
    $p(x, y) < \delta \Rightarrow \left| f_n(x) - f_n(y)\right| < \frac{\epsilon}{3} \; 
    \forall n \in \N$.
    Letting $n = n_k$, $p(x, y) < \delta \Rightarrow \left| g_k(x) - g_k(y) \right| < 
    \epsilon \; \forall k \in \N$.
    Since $X$ is compact, it is totally bounded, so $\exists \{x_{i}\}_{i=1}^{N} $ s.t. 
    $X \subseteq \bigcup_{i = 1}^N B^p(x_i, \delta)$
    Moreover, $\forall 1\leq i \leq N$ fixed, we know $\{g_{k}\}_{k=1}^{\infty} x_i \subseteq \R$
    converges becuase $\{g_{k}\}_{k=1}^{\infty} $ converges pointwise, so $\{g_{k}\}_{k=1}^{\infty}$
    is a Cauchy sequenece. So $\exists K_i > 0 $ s.t. $\forall k, \ell \geq K_i$, 
    $\left| g_k(x_i) - g_\ell(x_i) \right| \leq \frac{\epsilon}{3}$
    Let $K := \max_{1 \leq i \leq N}{K_i}$. Then, $\forall k, \ell \geq K$, we have
    $\left| g_k(x_i) - g_\ell(x_i) \right| \leq \frac{\epsilon}{3} \quad \forall 1 \leq i \leq N$.
    So $\forall x + X \subseteq \cup_{i = 1}^N B^p(x_i, \delta). \exists x_i $ s.t. 
    $p(x, x_i) < \delta$, and $\forall k, \ell > K,$
    \[
        \left| g_k(x) - g_\ell(x) \right| \leq 
        \left| g_k(x) - g_k(x_i) \right| + \left| g_k(x_i) - g_\ell(x_i) \right| + 
        \left| g_\ell(x_i) - g_\ell(x) \right| < \epsilon
    \]            
    $\Rightarrow \forall \epsilon > 0, \exists K > 0 $ s.t. $\forall k, \ell > K$,
    $\| g_k - g_\ell \|_{\infty} = \sup_{x \in X} \left| g_k(x) - g_\ell(x) \right| < \epsilon$, 
    so $\{g_{k}\}_{k=1}^{\infty} $ is uniformly Cauchy. Since $(X, p)$ is compact, $C(X)$ is complete, 
    so $\{g_{k}\}_{k=1}^{\infty} = \{f_{n_k}\}_{k=1}^{\infty} $ converges uniformly. 
    Since $f_{n_k} \to f$ pointwise in $X$, it must be that $f_{n_k} \to f$ uniformly, 
    and thus $f \in C(X)$.        
\end{proof}
\end{comment}
\begin{proof}
    Since $(X, p)$ is compact, it is thus separable. Also, uniform bounded/equicontinuous 
    implies pointwise bounded/equicontinuous. Therefore, by Arzelà-Ascoli lemma, 
    $\exists f: X \to \R$ and $\{f_{n_k}\}_{k=1}^{\infty}$ s.t. $f_{n_k} \to f$ pointwise in $X$.
    
    \noindent Now let $g_k := f_{n_k}$. 
    
    \medskip
    \noindent \textbf{Claim:} $\{g_{k}\}_{k=1}^{\infty}$ is uniformly Cauchy.\footnote{Cauchy sequence in $(C(X), \| \cdot \|_{\infty} )$}
    
    \medskip
    \noindent Fix $\epsilon > 0$. By uniform equicontinuity, $\exists \delta > 0$ s.t. 
    \[
        p(x, y) < \delta \implies \left| f_n(x) - f_n(y)\right| < \frac{\epsilon}{3} 
        \quad \forall n \in \N.
    \]
    Letting $n = n_k$, 
    \[
        p(x, y) < \delta \implies \left| g_k(x) - g_k(y) \right| < \epsilon \quad \forall k \in \N.
    \]
    Since $X$ is compact, it is totally bounded, so $\exists \{x_{i}\}_{i=1}^{N}$ s.t. 
    $X \subseteq \bigcup_{i = 1}^N B^p(x_i, \delta)$.
    
    Moreover, $\forall 1\leq i \leq N$ fixed, we know $\{g_{k}(x_i)\}_{k=1}^{\infty} \subseteq \R$ 
    converges because $\{g_{k}\}_{k=1}^{\infty}$ converges pointwise, so $\{g_{k}(x_i)\}_{k=1}^{\infty}$
    is a Cauchy sequence. So $\exists K_i > 0$ s.t. $\forall k, \ell \geq K_i$, 
    \[
        \left| g_k(x_i) - g_\ell(x_i) \right| \leq \frac{\epsilon}{3}.
    \]
    Let $K := \max_{1 \leq i \leq N}{K_i}$. Then, $\forall k, \ell \geq K$, we have
    \[
        \left| g_k(x_i) - g_\ell(x_i) \right| \leq \frac{\epsilon}{3} \quad \forall 1 \leq i \leq N.
    \]
    So $\forall x \in X \subseteq \bigcup_{i = 1}^N B^p(x_i, \delta)$, 
    $\exists x_i$ s.t. $p(x, x_i) < \delta$, and $\forall k, \ell > K$,
    \[
        \left| g_k(x) - g_\ell(x) \right| \leq \left| g_k(x) - g_k(x_i) \right| + 
        \left| g_k(x_i) - g_\ell(x_i) \right| + \left| g_\ell(x_i) - g_\ell(x) \right| < \epsilon.
    \]
    This implies $\forall \epsilon > 0, \exists K > 0$ s.t. $\forall k, \ell > K$,
    \[
        \| g_k - g_\ell \|_{\infty} = \sup_{x \in X} \left| g_k(x) - g_\ell(x) \right| < \epsilon,
    \]
    so $\{g_{k}\}_{k=1}^{\infty}$ is uniformly Cauchy. Since $(X, p)$ is compact, 
    $C(X)$ is complete, so $\{g_{k}\}_{k=1}^{\infty} = \{f_{n_k}\}_{k=1}^{\infty}$ 
    converges uniformly. Since $f_{n_k} \to f$ pointwise in $X$, it must be that $f_{n_k} \to f$ 
    uniformly, and thus $f \in C(X)$.        
\end{proof}

\begin{rem}
    How do we use the AA theorem? To extract convergent subsequence, which may give us convergence
    of the original sequence. 
\end{rem} 

\noindent \underline{Fact:} Let $\{f_{n}\}_{n=1}^{\infty} \subseteq C(X)$. If $\exists! \; f$ 
s.t. for every subsequence, $\exists $ a further subsequence $\{f_{n_{k_j}}\}_{j=1}^{\infty} $ s.t. 
$f_{n_{k_j}} \xrightarrow[j \to \infty]{} f$ uniformly, then $f_n \xrightarrow[n \to \infty]{} f $ 
uniformly.     

\noindent \underline{Typical Application}
\begin{itemize}
    \item Verify $\left\{ f_n \right\} $  satisfies hypothesis of AA;
    \item For every subseq $\left\{ f_{n_k} \right\} $ also satisfies hypothesis of AA; 
    \item Use AA to extract $\{f_{n_{k_j}}\}_{j=1}^{\infty} $ s.t. $f_{n_{k_j}} \to f$ uniformly on $X$.
    \item If you can show $f$ is unique, then $f_n \to f$ in $C(X)$.     
\end{itemize}

\begin{cor}
    Let $(X, p)$ be a compact metric space. Let $\mathcal{F} \subseteq C(X)$ be uniformly bounded and
    uniformly equicontinuous. Then, $\mathcal{F}$ is precompact in $(C(X), \| . \|_{\infty} )$.  
\end{cor}
\begin{proof}
    If $f$ is uniformly bounded and uniformly equicontinuous, then by the AA theorem, $\forall$ sequence
    $\{f_{n}\}_{n=1}^{\infty} \subseteq \mathcal{F}$, there is a subseq. $\{f_{n_j}\}_{j=1}^{\infty} $
    and $f \in C(X)$ s.t. $f_{n_j} \to f$ in $C(X)$. Note, $f$ may not be in $\mathcal{F}$. So
    $\mathcal{F}$ is precompact.          
\end{proof}

\begin{exmp}
    Let $M > 0$, $\mathcal{F} = \left\{ f \in C([a, b]) \cap C^1([a,b]) : \| f \|_{\infty} + 
    \| f' \|_{\infty} < M   \right\} $. $\mathcal{F}$ is uniformly bounded and uniformly equicontinuous. 
    So by AA then, for $\left\{ f_n \right\} \subseteq \mathcal{F}, \exists \{f_{n_k}\}_{k=1}^{\infty}$
    s.t. $f_{n_k} \to f$ uniformly. But, $f$ may not be in $C^1([a,b])$. 
    (So, $f$ may not be in $\mathcal{F}$  )    
\end{exmp}


\textbf{Extra stuff left in the assignment, go back and look at it.}

\subsection{Baire Category Theorem}

\begin{defn}[Hollow/Nowhere Dense]
    We say a set $E$ is \textit{hollow} if $\interior(E) = \emptyset$.
    \footnote{i.e. $E$ contains no nontrivial open sets} 
    We say 
    $E \subseteq X$ \textit{nowhere dense} if its closure is hollow, i.e.
    $\interior(\overline{E}) = \emptyset$. 
\end{defn}

\begin{rem}
    $E$ hollow $\iff E^c$ dense in $X$, since $\interior(E) = \emptyset \iff
    (\interior (E))^c = \overline{E^c} = X.$  
\end{rem} 

\underline{Goal:} When can we guarantee that 
\begin{itemize}
    \item a union of hollow sets is hollow?
    \item an intersection of dense sets is dense?
\end{itemize}

\begin{comment}
\begin{thm}
    (Baire Category Theorem): Let $(X, p)$ be a complete metric space.
    \newline
    Let $\left\{ F_n \right\}_{n =1}^{\infty} \subseteq X$ be a collection of closed hollow sets. 
    Then $\bigcup_{n = 1}^{\infty} F_n$ is hollow.   
    \newline
    Let $\left\{ \mathcal{O}_n \right\}_{n = 1}^{\infty} \subseteq X$  be a collection of 
    open dense sets. Then $\bigcap_{n = 1}^{\infty} \mathcal{O}_n$ is dense. 
\end{thm}
\begin{proof}
    $(2) \Rightarrow (1)$ by taking complements and using the previous remark, so we prove only $(2)$.
    \newline
    \underline{Claim:} Let $G = \cap_{n =1}^{\infty} \mathcal{O}_n$. Then $G$  is dense in $X$.
    \newline
    FIx $x \in X, r > 0$. $\forall n \in \N, \mathcal{O}_n$ is open and dense, so 
    $\exists y \in \mathcal{O}_n$ and $s > 0$ s.t. $B(x, r) \cap \mathcal{O}_n \supseteq B(y, 2s)
    \supseteq \overline{B(y, s)}$. Now we use this fact indictively in n. Let 
    $x_1 \in X, r_1 < \frac{1}{2}$ s.t. $\overline{B(x_1, r_1)} \subseteq B(x, r) \cap \mathcal{O}_1$.
    Let $x_2 \in X, r_2 < 2^{-2}$ s.t. $\overline{B(x_2, r_2)} \subseteq B(x_1, r_1) \cap \mathcal{O}_2$.
    Take $x_n \in X$, $r_n < 2^{-n}$ s.t. $\overline{B(x_n, r_n)} 
    \subseteq B(x_{n - 1}, r_{n -1}) \cap \mathcal{O}_n$. 
    \[\Rightarrow \overline{B(x_1, r_1)} \supseteq \overline{B(x_2, r_2)} 
    \supseteq \dots \supseteq \overline{B(x_n, r_n)} \supseteq \dots \],
    and $r_n \to 0$. Therefore $\{x_{n}\}_{n=1}^{\infty} $ is cauchy, and $(x, p)$ is complete, 
    so $\exists x_0 \in X$ s.t. $x_n \to x_0$. Thus, $x_0 = 
    \cap_{n = 1}^{\infty} \overline{B(x_n, r_n)}$.
    Since $x_0 \in \overline{B(x_n, r_n)} \subseteq \mathcal{O}_n \; \forall n$, and $x_0 \in 
    \overline{B(x_1, r_1)} \subseteq B(x, r) \Rightarrow x_0 \in G \cap B(x, r)$.
    $\Rightarrow G \cap B(x, r) \neq \emptyset \; \forall x \in X, \forall r > 0$.
    $\Rightarrow G$ is dense in $X$.
\end{proof}
\end{comment}

\begin{thm}[Baire Category Theorem]
    Let $(X, p)$ be a complete metric space.
    \begin{enumerate}
        \item Let $\left\{ F_n \right\}_{n =1}^{\infty} \subseteq X$ be a collection of closed hollow sets. 
        Then $\bigcup_{n = 1}^{\infty} F_n$ is hollow.   
        \item Let $\left\{ \mathcal{O}_n \right\}_{n = 1}^{\infty} \subseteq X$ be a collection of 
        open dense sets. Then $\bigcap_{n = 1}^{\infty} \mathcal{O}_n$ is dense. 
    \end{enumerate}
\end{thm}

\begin{proof}
    $(2) \Rightarrow (1)$ by taking complements and using the previous remark, so we prove only $(2)$.
    
    \medskip
    \noindent \textbf{Claim:} Let $G = \bigcap_{n =1}^{\infty} \mathcal{O}_n$. Then $G$ is dense in $X$.
    
    \medskip
    \noindent Fix $x \in X, r > 0$. $\forall n \in \N, \mathcal{O}_n$ is open and dense, so 
    $\exists y \in \mathcal{O}_n$ and $s > 0$ s.t. 
    \[
    B(x, r) \cap \mathcal{O}_n \supseteq B(y, 2s) \supseteq \overline{B(y, s)}.
    \]
    Now we use this fact inductively in $n$. Let 
    $x_1 \in X, r_1 < \frac{1}{2}$ s.t. $\overline{B(x_1, r_1)} \subseteq B(x, r) \cap \mathcal{O}_1$.
    Let $x_2 \in X, r_2 < 2^{-2}$ s.t. $\overline{B(x_2, r_2)} \subseteq B(x_1, r_1) \cap \mathcal{O}_2$.
    Repeating this process, take $x_n \in X$, $r_n < 2^{-n}$ s.t. $\overline{B(x_n, r_n)} 
    \subseteq B(x_{n - 1}, r_{n -1}) \cap \mathcal{O}_n$. 
    \[
    \Rightarrow \overline{B(x_1, r_1)} \supseteq \overline{B(x_2, r_2)} 
    \supseteq \dots \supseteq \overline{B(x_n, r_n)} \supseteq \dots,
    \]
    and $r_n \to 0$. Therefore $\{x_{n}\}_{n=1}^{\infty} $ is Cauchy, and $(X, p)$ is complete, 
    so $\exists x_0 \in X$ s.t. $x_n \to x_0$. Thus, 
    \[
    x_0 = \bigcap_{n = 1}^{\infty} \overline{B(x_n, r_n)}.
    \]
    Since $x_0 \in \overline{B(x_n, r_n)} \subseteq \mathcal{O}_n \; \forall n$, and $x_0 \in 
    \overline{B(x_1, r_1)} \subseteq B(x, r) \Rightarrow x_0 \in G \cap B(x, r)$.
    \[
    \Rightarrow G \cap B(x, r) \neq \emptyset \; \forall x \in X, \forall r > 0.
    \]
    $\Rightarrow G$ is dense in $X$.
\end{proof}

Another restatement of the Baire Category Theorem is as follows: 
If $(X, p)$ is complete, the countable union of nowhere dense sets is hollow.
\begin{proof}
    Let $\{E_{n}\}_{n=1}^{\infty} $ be nowhere dense sets. Then by BCT, $\cup_{n = 1}^{\infty}
    \overline{E_n}$ is hollow. It follows that $\cup_{n = 1}^{\infty} E_n \subseteq 
    \cup_{n = 1}^{\infty} \overline{E_n}$  so $\cup_{n = 1}^{\infty} E_n$ is also hollow. 
\end{proof}  

The main way we will use the Baire Category Theorem is the following:
\begin{cor}
    Let $(X, p)$ be complete. Suppose $\{F_{n}\}_{n=1}^{\infty} $ is a collection of closed sets.
    If $X = \cup_{n=1}^{\infty} F_n$, then $\exists n_0$ s.t. $\interior(F_{n_0}) \neq \emptyset$.   

\end{cor}
\begin{proof}
    If $\not\exists n_0$, then $F_n$ is hollow $\forall n$, so by BCT $X = \cup_{n =1}^{\infty} F_n$
    is hollow, but this is a contradiction because $X \subseteq X$ is open and nontrivial.     
\end{proof}

\noindent \underline{Baire Category Theorem Application:}

\begin{thm}
    Let $X \subseteq C(X)$ where $(X, p)$ is complete. Suppose $\mathcal{F}$ is pointwise bounded. 
    Then, $\exists $ non-empty open set $\mathcal{O} \subseteq X$ s.t. $\mathcal{F}$ is uniformly
    bounded on $\mathcal{O}$, i.e. $\exists M > 0$ s.t. \[
        \sup_{f \in \mathcal{F}} \sup_{x \in \mathcal{O}} \left|  f(x) \right| \leq M
    \] 
\end{thm}
\begin{proof}
    Let $E_n = \left\{ x \in X : \left| f(x) \right| \leq n \forall f \in \mathcal{F} \right\} =
    \bigcap_{f\in \mathcal{F}} \left\{ x \in X : \left| f(x) \right| \leq n  \right\}  $
    $\Rightarrow E_n$ is closed $\forall n$. Since $\mathcal{F}$ is pointwise bounded, 
    $\forall x \in X, \exists M_x > 0 $ s.t. $\sup_{ f \in \mathcal{F}} \left| f(x) \right|  \leq M_x$
    Thus, $\forall n$ s.t. $M_x \leq n$, then $x \in E_n$  $(\left| f(x) \right| \leq M_x \leq n)$.
    So, $X = \bigcup_{n=1}^{\infty} E_n$ and $E_n$  is closed. By corollary, $\exists n_0 $ s.t. 
    $\interior(E_{n_0}) \neq \emptyset$. So $\exists x_0 \in X, r$ s.t. $B^p(x_0, r) 
    \subseteq E_{n_0}$. Letting $\mathcal{O} = B^p(x_0, r)$, we have 
    $\sup_{x \in \mathcal{O}} \left| f(x) \right| \leq n_0 \quad \forall f \in \mathcal{F}$.  
\end{proof}

\begin{cor}
    Let $(X, p)$ be a complete metric space. Suppose $\{F_{n}\}_{n=1}^{\infty} $ is a collection
    of closed sets. Then $\bigcup_{n=1}^{\infty} \partial F_n$  is hollow.
\end{cor}
\begin{proof}
    \underline{Claim:} $\partial F_n$ is hollow $\forall n$. Suppose for contradiction that
    $\exists n$ s.t. $\interior (\partial F_n) \neq \emptyset$. Then $\exists x_0 \in
    \partial F_n, r > 0$ s.t. $B^p(x_0, r) \subseteq \partial F_n$. But then,
    \[
        B^p(x_0, r) \cap F_n^c = B^p(x_0, r) \cap \overline{F_n^c} = B^p(x_0, r) \cap 
        (F_n \cup \partial F_n)^c = B^p(x_0, r) \cap \partial F_n^c \cap F_n^c = \emptyset
    \]       
    and this contradicts $x_0 \in \partial F_n$ by defn. $\Rightarrow \partial F_n$ is hollow 
    $\forall n$.
    Furthermore, $\partial F_n$ is closed, since it contains all of its limit points by definition.
    Thus, by BCT, $\bigcup_{n=1}^{\infty} \partial F_n$ is hollow.    
\end{proof}

\noindent Now recall that in general, $\{f_{n}\}_{n=1}^{\infty} \subseteq C(X)$ and $f_n \to f$ 
pointwise, then $f$  is not necessarily continuous.  

\begin{comment}
\begin{thm}
    Let $(X, p)$ be complete. Let $\{f_{n}\}_{n=1}^{\infty} \subseteq C(X)$ s.t. $f_n \to f$
    pointwise in X. Then there is a dense subset $D \subseteq X$ where $\{f_{n}\}_{n=1}^{\infty} $
    is pointwise equicontinuous on D and $\forall x_0 \in D$, $f$ is continuous at $x_0$.         
\end{thm}

\begin{proof}
    Let $m, n \in \N$. Define 
    \[
    E(m , n) = \left\{ x \in X : \left| f_j(x) - f_k(x) \right| 
    \leq \frac{1}{m} \; \forall j,k \geq n \right\} = \bigcap_{j, k \geq n} \left\{ x \in X
    : \left| f_j(x) - f_k(x) \right| \leq \frac{1}{m} \right\}   
    \]

So $E(m ,n)$ is closed $\forall m, n$. Thus, by the corollary, $\bigcup_{m, n \in \N} 
\partial E(m ,n)$ is hollow.
$\Rightarrow D : = (\bigcup_{m , n \in \N} \partial E(m ,n))^c = \bigcap_{m, n \in \N}
\partial E(m ,n)^c$ is dense.

\textbf{Claim 1.} If $\exists x \in X, m, n \in \mathbb{N}$ s.t. $x \in D \cap E(m,n)$, then $x \in \interior(E(m,n))$.

If $x \in D$, then
\[
x \in \underbrace{\partial E(m,n)^c}_{\text{open}} = \interior(E(m,n)) \cup \exterior(E(m,n)).
\]
For the exterior term:
\begin{align*}
\exterior(E(m,n)) &= X \setminus (\interior(E(m,n)) \cup \partial E(m,n)) \\
&= X \setminus E(m,n) = E(m,n)^c.
\end{align*}
Since we also have $x \in E(m,n)$, this means $x \in \interior(E(m,n))$

\textbf{Claim 2.} $\{f_{n}\}_{n=1}^{\infty} $ is equicontinuous on D

Let $x_0 \in D$ and $\epsilon > 0$. Choose $m$ s.t. $\frac{1}{m} < \frac{\epsilon}{4}$. Since
$\{f_{n}\}_{n=1}^{\infty} $ converges, $\{f_{n}\}_{n=1}^{\infty} \subseteq \R$ is a Cauchy sequence.
So $\exists N$ s.t. $\forall j, k \geq N$,
\[
    \left| f_j(x_0) - f_k(x_0) \right| \leq \frac{1}{m}
\]         
This means $x_0 \in E(m, n) \cap D$, so by claim 1, $x_0 \in \interior(E(m ,n))$.
Let $B^p(x_0, r) \subseteq E(m ,N)$, so $\forall j, k \geq N$, $\forall x \in B(x_0 , r)$,
\[
    \left| f_j(x) - f_k(x) \right| \leq \frac{1}{m}.
\]       
Since $f_N$ is continuous at $x_0$, $\exists \delta_{x_0} > 0$ (Which WLOG we can choose $< r$ ), 
s.t. $\forall x \in B^p(x_0, \delta_{x_0})$,
\[
    \left| f_N(x) - f_n(x_0)  \right| \leq \frac{1}{m}
\]
so $\forall j \geq N, \forall x \in B^p(x_0, \delta_{x_0})$,
\[
    \left| f_j(x) - f_j(x_0) \right| \leq \left| f_j(x) - f_N(x) \right| + \left| f_N(x) - f_N(x_0) \right| 
    + \left| f_N(x_0) - f_j(x_0) \right| \leq \frac{3}{m} \leq \frac{3\epsilon}{4}
\]       

Since this holds $\forall j \geq N$, this implies that $\{f_{n}\}_{n=1}^{\infty} $ is equicnontinuous
at $x_0$. Furthermore, $\forall x \in B^p(x_0, \delta_{x_0})$, sending $j \to \infty$, we obtain
that $\forall x \in B^p(x_0, \delta_{x_0}), \left| f(x) - f(x_0) \right| \leq \frac{3\epsilon}{4}$,
so $f$ is continuois at $x_0 \in D$.        
\end{proof}
\end{comment}

\begin{thm}
    Let $(X, p)$ be complete. Let $\{f_{n}\}_{n=1}^{\infty} \subseteq C(X)$ s.t. $f_n \to f$ pointwise 
    in $X$. Then there is a dense subset $D \subseteq X$ where $\{f_{n}\}_{n=1}^{\infty}$ is pointwise 
    equicontinuous on $D$ and $\forall x_0 \in D$, $f$ is continuous at $x_0$.
\end{thm}

\begin{proof}
    Let $m, n \in \mathbb{N}$. Define 
    \[
    E(m , n) = \left\{ x \in X : \left| f_j(x) - f_k(x) \right| \leq \frac{1}{m} \; \forall j,k 
    \geq n \right\} = \bigcap_{j, k \geq n} \underbrace{\left\{ x \in X : \left| f_j(x) - f_k(x) 
    \right| \leq \frac{1}{m} \right\} }_{\text{closed, since }f_k, f_j \in C(X)}.
    \]
    So $E(m ,n)$ is closed $\forall m, n$. Thus, by the corollary, $\bigcup_{m, n \in \mathbb{N}} 
    \partial E(m ,n)$ is hollow. This implies that
    \[
    D := \left(\bigcup_{m , n \in \mathbb{N}} \partial E(m ,n)\right)^c = \bigcap_{m, n \in \mathbb{N}}
     \partial E(m ,n)^c \quad \text{is dense.}
    \]

    \noindent \textbf{Claim 1:} If $\exists x \in X, m, n \in \mathbb{N}$ s.t. $x \in D \cap E(m,n)$, 
    then $x \in \interior(E(m,n))$.

    \noindent If $x \in D$, then
    \[
    x \in \underbrace{\partial E(m,n)^c}_{\text{open}} = \interior(E(m,n)) \cup \exterior(E(m,n)).
    \]
    For the exterior term:
    \begin{align*}
    \exterior(E(m,n)) &= X \setminus (\interior(E(m,n)) \cup \partial E(m,n)) \\
    &= X \setminus E(m,n) = E(m,n)^c.
    \end{align*}
    Since we also have $x \in E(m,n)$, this means $x \in \interior(E(m,n))$.

    \noindent \textbf{Claim 2:} $\{f_{n}\}_{n=1}^{\infty}$ is equicontinuous on $D$.

    \noindent Let $x_0 \in D$ and $\epsilon > 0$. Choose $m$ s.t. $\frac{1}{m} < \frac{\epsilon}{4}$. 
    Since $\{f_{n}\}_{n=1}^{\infty}$ converges, $\{f_{n}(x_0)\}_{n=1}^{\infty} \subseteq \mathbb{R}$ 
    is a Cauchy sequence. So $\exists N$ s.t. $\forall j, k \geq N$,
    \[
        \left| f_j(x_0) - f_k(x_0) \right| \leq \frac{1}{m}.
    \]
    This means $x_0 \in E(m, n) \cap D$, so by Claim 1, $x_0 \in \interior(E(m ,n))$. 
    Let $B^p(x_0, r) \subseteq E(m ,N)$, so $\forall j, k \geq N$, $\forall x \in B(x_0 , r)$,
    \[
        \left| f_j(x) - f_k(x) \right| \leq \frac{1}{m}.
    \]
    Since $f_N$ is continuous at $x_0$, $\exists \delta_{x_0} > 0$ 
    (which WLOG we can choose $< r$), s.t. $\forall x \in B^p(x_0, \delta_{x_0})$,
    \[
        \left| f_N(x) - f_N(x_0) \right| \leq \frac{1}{m}.
    \]
    So $\forall j \geq N, \forall x \in B^p(x_0, \delta_{x_0})$,
    \begin{align*}
        \left| f_j(x) - f_j(x_0) \right| &\leq \left| f_j(x) - f_N(x) \right| + 
        \left| f_N(x) - f_N(x_0) \right| + \left| f_N(x_0) - f_j(x_0) \right| \leq \frac{3}{m} 
        \leq \frac{3\epsilon}{4}.
    \end{align*}
    Since this holds $\forall j \geq N$, this implies that $\{f_{n}\}_{n=1}^{\infty}$ is 
    equicontinuous at $x_0$. Furthermore, $\forall x \in B^p(x_0, \delta_{x_0})$, 
    sending $j \to \infty$, we obtain that $\forall x \in B^p(x_0, \delta_{x_0})$, 
    \newline $\left| f(x) - f(x_0) \right| \leq \frac{3\epsilon}{4}$, so $f$ is continuous at 
    $x_0 \in D$.
\end{proof}



\subsection{Topological Spaces}

\begin{defn}
    Let $X$ be a non empty set. A \textit{topology} $\mathcal{T}$ on $X$ is a collection of subsets of $X$,
    called open sets, such that
    \begin{itemize}
        \item $X, \emptyset \in \mathcal{T}$;
        \item If $\left\{ E_n \right\} \subseteq \mathcal{T}, \bigcap_{n=1}^{N} E_n \in \mathcal{T}$
        (closed under finite intersections);
        \item If $\left\{ E_n \right\} \subseteq \mathcal{T}, 
        \bigcup_{n = 1}^{\infty} E_n \in \mathcal{T}$ (closed under arbitrary unions);
    \end{itemize}
    If $x \in X$, a set $E \in \mathcal{T}$ containing x is called a \textit{neighborhood} of $x$.
    I like pie  
\end{defn}

\subsection{Separation, Countability, Separability}
\subsection{Continuity and Compactness}



\end{document} 
